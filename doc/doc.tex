\documentclass[12pt]{article}

\usepackage{polski}
\usepackage[utf8]{inputenc}

\makeatletter
\newcommand{\verbatimfont}[1]{\renewcommand{\verbatim@font}{\ttfamily#1}}
\makeatother

\begin{document}


\begin{center}
\huge
ALHE -- dokumentacja i sprawozdanie\\
Temat: SK.ALHE.6 \\

\bigskip

\LARGE
Patryk Pankiewicz, Maciej Turski \\
07.01.2019 
\end{center}

\newpage

\section{Temat projektu}

Masz 10 kart ponumerowanych od 1 do 10. Znajdź przy uzyciu Algorytmu Ewolucyjnego sposób na podział kart na dwie kupki w taki sposób, że suma kart na pierwszej kupce jest jak najbliższa wartości A, a suma kart na drugiej kupce jest jak najblizsza wartości B. Należy zastosować dodatkowo inny wybrany algorytm i porównać wyniki.

\section{Opis funkcjonalny}
Do rozwiązania problemu został użyty algorytm ewolucyjny, wykorzystujący krzyżowanie i mutacje. Celem algorytmu jest minimalizacja różnicy. Wartości kart są przechowywane jako list i nie ma potrzeby by była ona kopiowana dla każdego osobnika - jest on reprezentowany jako lista przynależności kart do stosów. Wykorzystane metody zostały wyszczególnione poniżej. 


\begin{itemize}
	\item{\textbf{Funkcja oceniająca} - musi być dostosowana do celu minimalizacji i zostały zaimplementowane jej nastepujące wersje:
		\begin{itemize}
			\item{suma modułów różnicy wartości kart na stosach} 
			\item{suma różnicy wartości kart na stosach podniesiona do potęgi \\ $>=2$ (jest ona parametrem algorytmu} 
		\end{itemize} 
	}
	\item{\textbf{Selekcja} osobników - zostały stworzone trzy wskazane w czasie wykładów z przedmiotu podejścia:
		\begin{itemize}
			\item{metoda turniejowa}
			\item{metoda progowa}
			\item{metoda proporcjonalna}
		\end{itemize}
	}
	\item{\textbf{Mutacja} - realizowana poprzez zmianę przynależności karty do stosu}
	\item{\textbf{Krzyżowanie} - zostały opracowane dwa rozwiązania:
		\begin{itemize}
			\item{dla każdej karty z listy losowany będzie z jednakowym prawdopodobieństwem osobnik z którego zostanie wzięta informacja gdzie ma przynależeć karta}
			\item{wylosowanie indeks lub indeksów względem których przynależność kart zostanie zmieniona na przeciwną}
		\end{itemize}			
		} 
\end{itemize}

\section{Opis interfejsu użytkownika}
Aplikacja posiada intefejs konsolowy. Argumenty aplikacji:

\begin{itemize}
	\item{-1 wartość - wartość oczekiwana dla pierwszego stosu, wymagane}
	\item{-2 wartość - wartość oczekiwana dla pierwszego stosu, wymagane}
	\item{rodzaj metody selekcji, wymagane - jeden argument z poniższych: 
		\begin{itemize}
			\item{-r wartość, selekcja proporcjonalna, wartość odpowiada liczbie \textit{beta}, która jest używana przy obliczniu wartości prawdopodobieństwa ze wzoru $exp(-beta*z_i)$, gdzie $z_i$ odpowiada wartości celu $i$-tego osobnika}
			\item{-t wartość, selekcja turniejowa, wartość odpowiada rozmiarowi szranek}
			\item{-h wartość, selekcja progowa, wartość odpowiada wartości progu}
		\end{itemize}
		}
	\item{-m wartość - wartość odpowiada prawdopodobieństwu mutacji, wymagane}
	\item{-v wartość - wybór funkcji celu, wartość odpowiada potędze do której zostanie podniesiona różnica wartość kart na obu stosach, wymagane }
	\item{rodzaj metody krzyżowania, wymagane - jeden argument z poniższych:
		\begin{itemize}
			\item{-u - wybór każdej właściciela dla każdej karty z osobna}
			\item{-k wartość - wybór $k$ punktów względem których zostanie zmieniona przynależność kart, wartość odpowiada liczbie $k$} 
		\end{itemize}
		}
	\item{-s wartość - wartość odpowiada wartości ziarna, jeśli nie podane zostanie ustawione na losową}
	\item{-c wartość - wartość odpowiada ilości kart, wymagane}
	\item{-a wartość - wartość odpowiada rozmiarowi populacji, wymagane}
	\item{-b wartość - wartość odpowiada prawdopodobieństwu mutacji, wymagane }
	\item{-i wartość - wartość odpowiada ilości iteracji, wymagane }
	\item{-f wartość - wartość odpowiada prefiksowi pliku wynikowego}
\end{itemize}

\section{Postać plików wynikowych}
Pliki wynikowe są plikami w formacie csv, z seperatorem '',''. Nazwa pliku wynikowego jest wartością podaną dla argumentu -f, będącego prefiksem, i argumentów programu, tak by ułatwić identyfikacje plików wynikowych. \\

Przykład nazwy pliku wynikowego: 
\verbatimfont{\footnotesize}
\begin{verbatim}
	wynik_-1_100_-2_100_-r_1_-m_0.09_-v_2_-u_-c_10_-a_30_-b_0.10_-i_200.csv
\end{verbatim}

Program został wywołany z następującymi argumentami:
\verbatimfont{\footnotesize}
\begin{verbatim}
-1 100 -2 100 -r 1 -m 0.09 -v 2 -u -c 10 -a 30 -b 0.10 -i 200 -f wynik
\end{verbatim}

Pozwala to w prosty sposób zidentyfikować wybrane argumenty. Program nie posiada żadnych dodatkowych pilków konfiguracyjny - wszystkie parametry są ustalane za pomocą argumentów. 

\bigskip

Dla każdej iteracji w wynikowym pliku csv zapisywane są następujące informacje:
\begin{itemize}
	\item{wartość funkcji celu dla najmniejszego osobnika}
	\item{wartość mediany populacji}
	\item{wartość średnia populacji}
	\item{wartość odchylenia standardowego populacji}
\end{itemize}

\section{Opis implementacji}
Algorytm został zaimplementowany w języku C++ w standardzie C++14 z użyciem wyłącznie bibliotek dostepnych w standardzie. Program został on zaprojektowny w sposób modularny z wykorzystaniem klas abstrakcyjnych. Główne klasy algorytmu:

\begin{itemize}
	\item{\textbf{Mutation} - klasa odpowiedzialna za przeprowadzenie mutacji z zadanym prawdopodobieństwem}
	\item{\textbf{CrossoverAlgorithm} - klasa abstrakcyjna będąca klasą bazową dla algorytmów krzyżowania}
	\item{\textbf{ScoringFunction} - klasa, która pozwala na obliczenie zadanej funkcji celu}
	\item{\textbf{SelectionAlgorithm} - klasa abstrakcyjna będąca klasą bazową dla algorytmów selekcji}
	\item{\textbf{EvolutionaryAlgorithm} - główna klasa algorytmu}
	\item{\textbf{CSVFileWriter} - klasa opowiadająca za zapis wyników do pliku csv} 
\end{itemize}

Dodatkowo do zautomatyzowania testowania i generacji wykresów z danych wynikowych został użyty język Python. Zostały użyte nastepujące moduły i biblioteki:

\begin{itemize}
	\item{pandas\footnote{https://pandas.pydata.org/}, biblioteka umożlwiająca łatwą obsługę plików o rozszerzeniu csv}
	\item{numPy\footnote{http://www.numpy.org/}, biblioteka obliczeniowa, tu użyta do obsługi tablic i prostych algorytmów} 
	\item{matplotlib\footnote{https://matplotlib.org/}, biblioteka do generowania wykresów}
	\item{subprocess\footnote{https://docs.python.org/3/library/subprocess.html}, moduł umożlwiający wywołanie plików wykonywalnych} 	
\end{itemize}


\section{Badania}


\end{document}